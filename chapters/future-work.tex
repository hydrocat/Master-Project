\chapter{Future Work}


\section{Code Re-structure}
\subsection{Administrative Resources}
Currently, resource filtering is implemented by having a small set of controllers that are eligible to handle non-administrative users however this led to some confusion because its response structure does not follow the \gls{HATEOAS} convention.

Some re-structuring could be done at the repository level by configuring the authorization mechanism to filter \gls{HTTP} methods based on the current user's requesting role.

Ultimately, all resources should follow the \gls{HATEOAS} pattern so that the complexity of the front-end application is lower.

\subsection{PermissionTree's cyclic reference}
In the back-end, the \texttt{PermissionTree} class is implemented with a reference to its parent. In the end this attribute did not contribute to anything useful to the architecture and instead induced the creation of custom controllers and View-Models that brought inconsistencies in the angular application.

\section{Bulk information manager}
One of the requirements found in the proposal is the possibility to insert validated queries individually or in a group. The current system only supports inserting single, not validated  queries.

\section{Testing and User Validation}
Unit testing was done during the early stages of development but unfortunately they required too many resources. The multi-database support required all database instances to be running during the test execution. As the project development sped up, tests were ditched. Therefore the implementation of a proper test suit that validates this project's requirements is considered valuable.

The introduction of the system in the proposed context was neither executed nor evaluated, leading to no conclusion whether it actually decreases the amount of times the \gls{IT} department is interrupted. Taking actual measurements before and after deployment would answer this question.

The developed \gls{UI}'s usability was not evaluated. Confusing interfaces lead to more interruptions to the \gls{IT} department, therefore usability tests should be done to assess its intuitiveness and ease-of-use.

\section{Parameterization}
The proposal states that this tool should enable \gls{SQL} queries to be parameterized, which would allow the system to meet the specific needs of each user. Unfortunately this field was not developed at all and queries can only be run as they were inserted.

