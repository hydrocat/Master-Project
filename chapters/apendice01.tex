\chapter{Proposta Original do Projeto}\label{apendice1}
\section{Proposta nº 2}

Pretende-se desenhar e construir de raíz um sistema de “business inteligence” aplicado à gestão letiva. Sabemos bem o valor e a importância que a informação tem hoje em dia para quem gere instituições e as mais valias que as ferramentas de análise de dados trazem para a tomada de medidas e decisões. O IPB tem já uma base de dados centralizada à qual é aplicado diariamente um grande número de queries em SQL para os mais diversos fins. Pretende-se abrir o acesso a esta informação de forma criteriosa mas sem implicar a escrita manual de queries muitas delas com mais de 30 linhas de código.

O sistema seria pré-alimentado com “clusters” de queries mas teria uma característica evolutiva que daria a possibilidade de introduzir de forma fácil, suportada e validada novas queries ou novos grupos de queries, dependendo do perfil do utilizador.

As palavras chave serão a reutilização e a parametrização automática desses grupos de queries suportadas pela geração automática de interfaces web de pesquisa de informação, tendo por base o tipo de query a implementar.

Trata-se da disponibilização de um sistema de consulta inteligente no sentido em que se adapta às necessidades e ao perfil de cada utilizador. O resultado final serão sempre tabelas exportáveis para os mais variados formatos.
