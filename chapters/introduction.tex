\chapter{Introduction}\label{chapter:intro}
As companies and institutions develop, they tend to implement and depend on digital systems~\cite{ibm}. Generally speaking, these solutions involve the deployment of a persistence mechanism such as a relational database or a directory service, \gls{RDBMS} being the most common~\cite{apachedp}. Once  made available, they are frequently updated with new information that, if not properly processed and made understandable, does not generate any meaningful insights.

It is not a hard task to give access to the raw information contained within a \gls{RDBMS} but because of how it is usually split in logical relations to avoid unnecessary data repetition~\cite[Part~$\mathrm{V}$]{vaquinha}, technical knowledge is required to harness its potential into a tangible understanding that can help in the decision-making process. 

A good system would enable anyone, expert in computational systems or not, to correlate and extract any information held in their institution. However such system would have to deal with many corner cases and thus an approximation that is able to handle the most frequent cases is useful even if it requires some amount of manual maintenance.

\section{Context}\label{context}
With more than 8,500 students and professors, \gls{IPB} is composed of 5 schools that span diverse fields of study including but not limited to Education, Administration, Chemistry, Health, Tourism, Biology and Engineering.

Towards the beginning and the end of the semester, during student registration time, professors often need some insights on their educational affairs. To do so, they reach out to \gls{IT} department inquiring about an specific need, the technician then stop his current task, write a \gls{SQL} script, run it on the institution's \gls{RDBMS} and email back the results as an spreadsheet file.

Over time the technician has built a list of about 40 common requests and their respective scripts so that this process takes less of his time, enabling him to continue developing \gls{IPB}'s in-house software. However, interruptions still happen often enough that an automated system is still needed.

\section{Objective}\label{objective}

The objective of this project is to develop \gls{Yabi} a web platform that exposes \gls{SQL} scripts to \gls{IPB}'s employees in a convenient way that does not require expert knowledge of the underlying system architecture and eases the \gls{IT} department workload during the institution's critical moments.

Such platform will be maintained by an administrator that is responsible for registering the desired \gls{SQL} scripts and based on their role in the institution, professors and other employees are presented to a list of queries coupled with meaningful title and description in which they can run, see the resulting table and download an spreadsheet file.

Following is a translation of the original proposal found in Appendix~\ref{apendice1}:
\begin{displayquote}
  Proposal nº 2\\
  To architect and construct from scratch a ``business intelligence'' system with emphasis on education management.
  Now a days we know how valuable and important information is for those who manage institutions and the impact that data analysis tools have in the decision-making process.
  At the time writing, \gls{IPB} is already provided of a centralized database through which a large number of \gls{SQL} of many diverse purposes queries are run.
  The intention is that, based on certain criteria, provide access to information without having to manually write queries that are often more than 30 lines long.

  The proposed system would be fed with ``clusters'' of queries and provide a way to easily insert and validate individual or group of queries depending on the currently logged-in user's profile.

  Keywords are reuse and automatic parameterization of queries that are supported by an automatic web search interface based on the current query.

  All in all, it is about deploying a intelligent search system that is able to adapt to the necessities and profile of each user. The end result will always be tables of data that can be exported to many different formats
\end{displayquote}

\section{Textual Conventions}

Throughout this document, some words and terms were made to look different from standard text to help convey the context and indicate the class in which the subject is part of. 

\begin{description}
\item[\texttt{Typewriter}] This is used to reference pieces of code such as data types, class names, method invocations that are written with it's class name and abstract parts of a system. E.g. \texttt{Authentication}, \texttt{SqlQueryController}, \texttt{DatabaseReader.runQuery}, \texttt{varchar}.
\item[\textit{Italic}] Refers to function and method names. E.g. \textit{authenticate}, \textit{runQuery}.
\item[\textbf{Bold}] Indicates resource paths. E.g. \textbf{/user}, \textbf{/PermissionTrees}, \textbf{/runQuery/\{id\}}.
\item[\textsc{Small Capital}] Indicates HTTP verbs. E.g. \textsc{delete}, \textsc{get}.
\end{description}

\section{Document Structure}

This document is divided in six chapters. Chapter~\ref{cha:concepts} introduces the tools and concepts employed in this application, grouping them by whether they were used in the front-end, back-end or during development.

With an analysis of this project's proposal and the context it supposed to be used, Chapter~\ref{chap:project} elicits requirements, use cases and entities that are found to compose the system. Chapter~\ref{cha:implementation} then goes through the implementation of those entities and use cases into front-end and back-end applications alongside the tools used to mimic the production environment locally.

Chapter~\ref{conclusion} concludes this document with an overview of what was done and comments its current state and lastly, Chapter~\ref{chap:future} presents a few ways in which this project could be further developed.