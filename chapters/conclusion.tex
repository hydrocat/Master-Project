\chapter{Conclusion}\label{conclusion}

With the growing adoption of digital processes by companies and institutions, the access to information becomes less available to the general public and more focused to experts in the field. These experts then end up mediating the interaction between those who are part of the decision-making process and the information storage.

Being in this situation, \gls{IPB}'s \gls{IT} department is often interrupted form their daily tasks to handle the most diverse inquiry and questions about the institution's data base.

The present system is designed to aid everyone that takes part in this process to access information in an efficient and organized manner without the need for much technical knowledge, which in turn lowers the amount of daily interruptions in the \gls{IT} department.
% end of recapitulacao
% capacidades
It currently achieves this by providing a web interface in which the institution members can login, choose one of the many inquiries registered by the \gls{IT} department and download its results.

% o que ficou ruim
In the end, some functionalities were let out of this implementation, with the most missed one is the ability for the users to tweak their inquiries to their specific needs.
% O que ficou bom
In spite of this the final system accomplishes the main task of providing users with the most used inquiries given that the \gls{IT} department registers them. Not only providing the inquiries, it also supplies the necessary tools to manage users, permissions and remote databases, all through a web interface.