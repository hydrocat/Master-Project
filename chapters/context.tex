\chapter{Context}\label{context}

\todo{Isso parece mais como uma introducao}
As companies and institutions develop, they tend to implement and depend on digital systems~\cite{ibm}. Generally speaking, solutions involve the deployment of a information center such as a relational database or a directory service, \gls{RDBMS} being the most common~\cite{apachedp}. Once the information center is made available, it is frequently updated with new information that, if not properly processed and made available, does not generate any meaningful insight.

It is not a hard task to give access to the raw information contained within a \gls{RDBMS} but because of how it is split in logical relations to avoid unnecessary data repetition~\cite[Part~$\mathrm{V}$]{vaquinha}, technical knowledge is required to harness its potential into a tangible understanding that can help in the decision-making process. 

A good system would enable anyone, expert in computational systems or not, to co-relate and extract the information held in their institution, however this is too broad of a scope. Therefore, an approximation of such system that is able to give meaningful insights and handle the most frequent cases is considered good enough even if it requires some manual work by an administrator.

% Actual context in which the application will run `Enquadramento`

\todo{Acho que esse eh um contexto melhor}

With more than 8,500 students and professors, \gls{IPB} is composed of 5 schools that span diverse fields of study including but not limited to Education, Administration, Chemistry, Health, Tourism, Biology and Engineering.

Towards the beginning and the end of the semester and during student registration time, professors often need some insights on their educational affairs. To do so, they get reach out to the Academic System department and request in broad terms what they need, the technician then stop his current task, write a \gls{SQL} script, run it and email back the results as an spreadsheet file.

With time, the technician built a list of about 40 common requests and their respective scripts so that this process takes less of his time, which in turn enables him to further develop \gls{IPB}'s in-house software. However, interruptions still happen often enough that an automated system is still needed.

\todo{acho que isso nao precisa vvv Talvez vira intrtoducao}

On a daily basis, a lot new information is generated from its many In-house and third-party software. Their custom administrative portal, \texttt{On-Line}\footnote{https://apps2.ipb.pt/online}, alone is able to handle many aspects of an student's interaction with the institution, for example, it can manage student balance, internship application, document request, submission of final thesis and reports, reserving meals and managing an electric bicycle rental subsystem called \texttt{IPBike}\footnote{http://ipbike.ipb.pt}. More specific features are available in function of user roles so that professors, researchers, employees, and course coordinators spend less time and resources into administrative tasks.



